%% Преамбула TeX-файла

% 1. Стиль и язык
\documentclass[utf8x, 12pt]{G7-32} % Стиль (по умолчанию будет 14pt)


% Остальные стандартные настройки убраны в preamble.inc.tex.
\sloppy

% Настройки стиля ГОСТ 7-32
% Для начала определяем, хотим мы или нет, чтобы рисунки и таблицы нумеровались в пределах раздела, или нам нужна сквозная нумерация.
\EqInChapter % формулы будут нумероваться в пределах раздела
\TableInChapter % таблицы будут нумероваться в пределах раздела
\PicInChapter % рисунки будут нумероваться в пределах раздела

% Добавляем гипертекстовое оглавление в PDF
\usepackage[
bookmarks=true, colorlinks=true, unicode=true,
urlcolor=black,linkcolor=black, anchorcolor=black,
citecolor=black, menucolor=black, filecolor=black,
]{hyperref}

% Изменение начертания шрифта --- после чего выглядит таймсоподобно.
% apt-get install scalable-cyrfonts-tex

% \IfFileExists{cyrtimes.sty}
%     {
%         \usepackage{cyrtimespatched}
%     }
%     {
%         % А если Times нету, то будет CM...
%     }

\usepackage{graphicx}   % Пакет для включения рисунков
\DeclareGraphicsExtensions{.jpg,.pdf,.png}
% С такими оно полями оно работает по-умолчанию:
% \RequirePackage[left=20mm,right=10mm,top=20mm,bottom=20mm,headsep=0pt]{geometry}
% Если вас тошнит от поля в 10мм --- увеличивайте до 20-ти, ну и про переплёт не забывайте:
\geometry{right=20mm}
\geometry{left=30mm}



% Произвольная нумерация списков.
\usepackage{enumerate}

\setcounter{tocdepth}{1} %Подробность оглавления
%4 это chapter, section, subsection, subsubsection и paragraph
%3 это chapter, section, subsection и subsubsection
%2 это chapter, section, и subsection
%1 это chapter и section
%0 это chapter.

\usepackage{placeins}
\setlength{\parindent}{15mm}
\graphicspath{{pictures/}}
%\usepackage{pscyr}
%  \renewcommand{\rmdefault}{ftm} % шрифт таймс                  %% загружает пакет многоязыковой вёрстки
\begin{document}

\frontmatter % выключает нумерацию ВСЕГО; здесь начинаются ненумерованные главы: реферат, введение, глоссарий, сокращения и прочее.
\begin{center} 

\large Московская Алкогольная Империя (Национальный исследовательский университет)\\[5.5cm] 

\huge Реферат \\[0.6cm] % название работы, затем отступ 0,6см
\large на тему:  Расчет красоты девочек в интернете\\[3.7cm]


\end{center} 

\begin{flushright}
Выполнил: студент гр. Т12О-302Б-20 \\
Иван Мамонтов \\
\end{flushright}


\vfill 

\begin{center} 
\large Москва 2023
\end{center} 

\thispagestyle{empty}

\thispagestyle{empty}
\setcounter{page}{0}
\tableofcontents
\clearpage


\Introduction

В современном мире, когда для решения большинства задач по знакомству, применяются различные ПО с доступом во всемирную сеть Интернет, 

\mainmatter

\chapter{Краткий обзор САПР систем}
Системы автоматизированного проектировния - комплексное обозначение набора ПО, позволяющего 
чертить, проектировать, просматривать, редактировать, рассчитывать различные детали, машины и механизмы. 
Конкретный функционал той или иной системы может сильно отличаться, ровно как и реализация взаимодействия с пользователем.
\par
САПР системы, как и языки программирования, разделяются на низкоуровневые, среднеуровневые и высокоуровневые в зависимости от того, насколько широкий функционал в них заложен. \\
Так, наглядным примером низкоуровневой системы будет ПО "EMPro", созданное конкретно для электромагнитного проектирования. У таких систем имеются опрределённые плюсы:
\begin{itemize}
	\item Относительно невысокая стоимость ПО
	\item Наиболее тщательная продуманность области, затрагиваемой системой
	\item Поддержка пользовательских модификаций и простота их создания
	\item Отсутствие надобности приобретать дополнения для реализации требуемых возможностей ПО
\end{itemize}
\par
К среднеуровным САПР можно отнести Autodesk Inventor - она позволяет проектировать модели, разрабатывать изделия из листового металла, создавать литейную оснастку, электрические и трубопроводные сети, производить расчеты и визуализации по построенным моделям. Такая система предоставляет относительно широкий функционал для проектирования деталей и машин, однако функционал в спецефичных областях достаточно узок и покрывает минимальные нужды. \par
Высокоуровневые САПР являются синтезом среднеуровневых и низкоуровневых ПО, позволяющих проводить весь цикл проектирования в одной среде. Они намеренно охватывают все возможные области проектирования, зачастую вынося отдельные функции из базовой версии.\par
Преимущества таких систем:
\begin{itemize}
	\item Удобство использования одной среды разработки для всех областей
	\item Гибкость системы в зависимости от подключаемых модулей
	\item Высокая продуманность и дружелюбность интерфейся к пользователя ввиду сложности разработки программы
\end{itemize}


\backmatter %% Здесь заканчивается нумерованная часть документа и начинаются ссылки и
            %% заключение

\Conclusion % заключение к отчёту

Текст заключения


\nocite{*}
\bibliographystyle{gost780u}
\bibliography{0-main}


\appendix   % Тут идут приложения

\chapter{Первое Приложение}

\end{document}
