\chapter{Краткий обзор САПР систем}
Системы автоматизированного проектировния - комплексное обозначение набора ПО, позволяющего 
чертить, проектировать, просматривать, редактировать, рассчитывать различные детали, машины и механизмы. 
Конкретный функционал той или иной системы может сильно отличаться, ровно как и реализация взаимодействия с пользователем.
\par
САПР системы, как и языки программирования, разделяются на низкоуровневые, среднеуровневые и высокоуровневые в зависимости от того, насколько широкий функционал в них заложен. \\
Так, наглядным примером низкоуровневой системы будет ПО "EMPro", созданное конкретно для электромагнитного проектирования. У таких систем имеются опрределённые плюсы:
\begin{itemize}
	\item Относительно невысокая стоимость ПО
	\item Наиболее тщательная продуманность области, затрагиваемой системой
	\item Поддержка пользовательских модификаций и простота их создания
	\item Отсутствие надобности приобретать дополнения для реализации требуемых возможностей ПО
\end{itemize}
\par
К среднеуровным САПР можно отнести Autodesk Inventor - она позволяет проектировать модели, разрабатывать изделия из листового металла, создавать литейную оснастку, электрические и трубопроводные сети, производить расчеты и визуализации по построенным моделям. Такая система предоставляет относительно широкий функционал для проектирования деталей и машин, однако функционал в спецефичных областях достаточно узок и покрывает минимальные нужды. \par
Высокоуровневые САПР являются синтезом среднеуровневых и низкоуровневых ПО, позволяющих проводить весь цикл проектирования в одной среде. Они намеренно охватывают все возможные области проектирования, зачастую вынося отдельные функции из базовой версии.\par
Преимущества таких систем:
\begin{itemize}
	\item Удобство использования одной среды разработки для всех областей
	\item Гибкость системы в зависимости от подключаемых модулей
	\item Высокая продуманность и дружелюбность интерфейся к пользователя ввиду сложности разработки программы
\end{itemize}
