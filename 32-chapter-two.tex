\chapter{Выполнение задания}
\label{cha:ch_2}
Задание для курсовой работы наглядно демонстрирует преимущества САПР. В этой главе будет приведено полное описание процесса разаботки детали.
\FloatBarrier
\begin{figure}[ht]
    \centering
    \includegraphics[width = 0.8\linewidth]{79}
\end{figure}
\FloatBarrier

\section{Задание для курсовой работы}
В качестве источника всей информации о детали для построения дан чертёж, приведённый на рисунке \ref{Chertezh}
\FloatBarrier
\begin{figure}[ht]
    \centering
    \includegraphics[width = 0.8\linewidth]{12.jpg}
\end{figure}
\label{Chertezh}
\FloatBarrier
Так как на чертеже имеются множественные ошибки, деталь во многом построена
из принципа соответсвия эскизу, а не размерам. Все ошибки, отмеченные во время выполнения задания,риведены в приложении А. 
Все элементы, размеры которых подверглись изменениям обозначены либо красным (выданый размер неверен), либо зелёным (размер не дан)
\par
Ниже будет приведён процесс построения детали.
\FloatBarrier
\section{Наложение эскизов}
Для удобства работы с деталью на базовые плоскости стоит наложить изображения эскиза. Референсным значением для размеров эскиза выбрана величина 250 и подобрана с помощью инструмента масштабирования (для первого эскиза. В параметры последующих вставляются размеры первого).
Также изображения, для удобства, наложены с прозрачностью по цвету фона чертежа.
\begin{figure}[ht]
    \centering
    \includegraphics[width = 0.8\linewidth]{2.png}
\end{figure}
\FloatBarrier

\section{Построение полуцилиндров}
Исходной точкой будет выбрана точка, совпадающая с центром окружности, образующий левый цилиндр. На базовой плоскости "Спереди" построим эскиз обоих полуцилиндров.
Размеры определяем по чертежу до полного определения(эскиз из синего становится чёрным). 
\FloatBarrier
\begin{figure}[ht]
    \centering
    \includegraphics[width = 0.8\linewidth]{цилиндры.png}
\end{figure}
\FloatBarrier

Используем инструмент "Вытянутая бобышка"
\FloatBarrier
\begin{figure}[ht]
    \centering
    \includegraphics[width = 0.8\linewidth]{3.png}
\end{figure}
\FloatBarrier

Добавляем литейный уклон с помощью инструмента "Уклон" для обоих бобышек по отдельности. 
В качестве нейтральной выбираем дальнюю от базовой плоскости грань, под уклон выбираем только внешнюю поверхность.
\FloatBarrier
\begin{figure}[ht]
    \centering
    \includegraphics[width = 0.8\linewidth]{4.png}
\end{figure}
\FloatBarrier

Отражаем тело относительно базовой плоскости с помощью инструмента "Зеркальное отражение"
\FloatBarrier
\begin{figure}[ht]
    \centering
    \includegraphics[width = 0.8\linewidth]{6.png}
\end{figure}
\FloatBarrier

\section{Создание толстого основания}
На плоскости "Сверху" создаём эскиз и на нём по заданным размерам 
строим контур основания с учётом литейного уклона.
\FloatBarrier
\begin{figure}[ht]
    \centering
    \includegraphics[width = 0.8\linewidth]{8.png}
\end{figure}
\FloatBarrier

Вытягиваем бобышку с уклоном наружу, от смещения но 80мм до базовой плоскости.
\FloatBarrier
\begin{figure}[ht]
    \centering
    \includegraphics[width = 0.8\linewidth]{10.png}
\end{figure}
\FloatBarrier

\section{Создание полой крышки}
На виде "Спереди" строим эскиз внешнего контура крышки (зелёная кривая соответствует 
положению кривой на виде сверху, поэтому отклоняется от эскиза) 
\FloatBarrier
\begin{figure}[ht]
    \centering
    \includegraphics[width = 0.8\linewidth]{11.png}
\end{figure}
\FloatBarrier

Вытягиваем бобышку на расстояние, указанное на чертеже. 
\FloatBarrier
\begin{figure}[ht]
    \centering
    \includegraphics[width = 0.8\linewidth]{12.png}
\end{figure}
\FloatBarrier

Аналогично прошлой операции строим эскиз внутреннего контура крышки, но вместо бобышки вытягиваем вырез. 
\FloatBarrier
\begin{figure}[ht]
    \centering
    \includegraphics[width = 0.8\linewidth]{15.png}
\end{figure}
\FloatBarrier

\section{Создание третьего полуцилиндра}
Строим на плоскости "Справа" эскиз с полуцилиндром. 
\FloatBarrier
\begin{figure}[ht]
    \centering
    \includegraphics[width = 0.8\linewidth]{16.png}
\end{figure}
\FloatBarrier

Вытягиваем бобышку от плоскости смещения по граничному условию "До следующего".
\FloatBarrier
\begin{figure}[ht]
    \centering
    \includegraphics[width = 0.8\linewidth]{17.png}
\end{figure}
\FloatBarrier

Создаём на плоскости "Спереди" эскиз повёрнутого выреза на трубе. 
\FloatBarrier
\begin{figure}[ht]
    \centering
    \includegraphics[width = 0.8\linewidth]{18.png}
\end{figure}
\FloatBarrier

Создаём вырез с помощью команды "Повёрнутый вырез" вокруг оси (Справочная линия), построенной на эскизе. 
\FloatBarrier
\begin{figure}[ht]
    \centering
    \includegraphics[width = 0.8\linewidth]{19.png}
\end{figure}
\FloatBarrier

\section{Создание тонкого основания}
На плоскости "Сверху" создаём эскиз контура основания. Оканчивается контур в любом месте внутри ушек справа.
\FloatBarrier
\begin{figure}[ht]
    \centering
    \includegraphics[width = 0.8\linewidth]{20.png}
\end{figure}
\FloatBarrier

Вытягиваем бобышку из эскиза. 
\FloatBarrier
\begin{figure}[ht]
    \centering
    \includegraphics[width = 0.8\linewidth]{21.png}
\end{figure}
\FloatBarrier

\section{Вырезание отверстий}
Аналогично прошлым операциям строим эскиз вырезов и вытягиваем их насквозь в оба направления. 
\FloatBarrier
\begin{figure}[ht]
    \centering
    \includegraphics[width = 0.8\linewidth]{23.png}
\end{figure}
\FloatBarrier

\section{Создание ушек на трубе}
Аналогично прошлым операциям создаём эскиз на плоскости "Сверху" и выягиваем от смещения бобышки с уклоном наружу. 
\FloatBarrier
\begin{figure}[ht]
    \centering
    \includegraphics[width = 0.8\linewidth]{25.png}
\end{figure}
\FloatBarrier

